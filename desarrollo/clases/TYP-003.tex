% Tipo que representa PUNTOS DE RECOGIDA

\begin{longtable}{p{2.5cm}  p{14cm}}
\caption*{\textbf{TYP-003: puntoRecogida}} \\
\hline
\textbf{Versión} & 1.0 (01/04/2014) \\
\textbf{Fuentes} & Apresa21 \\
\textbf{Descripción} & Este tipo de objetos representa  \textit{los puntos de recogida donde se depositarán los muebles y enseres para su posterior recogida.} \\
\textbf{Supertipo} & - \\
\textbf{Subtipo} & -  \\
\textbf{Comentarios} &- \\
\end{longtable}

% Atributos
\begin{longtable}{p{3cm}  p{12cm}}
\hline
\textbf{Atributo variable} & puntoRecogida::calle \\
\textbf{Descripción} & Este atributo representa  \textit{el nombre de la calle donde esta ubicado el punto de recogida.} \\
\textbf{Tipo} & \textit{String} \\
\textbf{Comentarios} & - \\
\end{longtable}

\begin{longtable}{p{3cm}  p{12cm}}
\hline
\textbf{Atributo variable} & puntoRecogida::latitud \\
\textbf{Descripción} & Este atributo representa  \textit{la latitud de la coordenada en la que se encuentra el punto de recogida} \\
\textbf{Tipo} & \textit{Double} \\
\textbf{Comentarios} & - \\
\end{longtable}

\begin{longtable}{p{3cm}  p{12cm}}
\hline
\textbf{Atributo variable} & puntoRecogida::longitud \\
\textbf{Descripción} & Este atributo representa  \textit{la longitud de la coordenada en la que se encuentra el punto de recogida} \\
\textbf{Tipo} & \textit{Double} \\
\textbf{Comentarios} & - \\
\end{longtable}

% Invariantes
\begin{longtable}{p{3cm}  p{12cm}}
\hline
\textbf{Expresión de invariante} & SolicitudRecogida::UnicidadLatitudLongitud \\
\textbf{Descripción} &  \textit{No existirán dos puntos de recogida ubicados en la misma coordenada, es decir que posean la misma latitud y longitud. } \\
\textbf{Comentarios} & -\\
\end{longtable}