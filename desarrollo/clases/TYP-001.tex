% Tipo que representa USUARIOS

\begin{longtable}{p{2.5cm}  p{14cm}}
\caption*{\textbf{TYP-001: Usuario}} \\
\hline
\textbf{Versión} & 1.0 (31/03/2014) \\
\textbf{Fuentes} & Apresa21 \\
% Descripción del Tipo
\textbf{Descripción} & Este tipo de objetos representa  \textit{los usuarios que solicitan recogida de muebles o enseres} \\
\textbf{Supertipo} & - \\
\textbf{Subtipo} & -  \\
\textbf{Comentarios} &- \\
\end{longtable}

\begin{longtable}{p{3cm}  p{12cm}}
\hline
\textbf{Atributo variable} & Usuario::teléfono \\
% Descripción del Atributo
\textbf{Descripción} & Este atributo representa  \textit{el número de teléfono del usuario} \\
\textbf{Tipo} & \textit{String} \\
\textbf{Comentarios} & Este campo se empleará para identificar al usuario. El teléfono es un teléfono de móvil. En caso de que haya alguna incidencia se contactará a este número. \\
\end{longtable}

\begin{longtable}{p{3cm}  p{12cm}}
\hline
\textbf{Atributo variable} & Usuario::Nombre \\
% Descripción del Atributo
\textbf{Descripción} & Este atributo representa  \textit{el nombre del solicitante del servicio de recogida y del titular del teléfono móvil.} \\
\textbf{Tipo} & \textit{String} \\
\textbf{Comentarios} &- \\
\end{longtable}

\begin{longtable}{p{3cm}  p{12cm}}
\hline
\textbf{Atributo variable} & Usuario::Apellidos \\
% Descripción 
\textbf{Descripción} & Este atributo representa  \textit{los apellidos del solicitante del servicio de recogida y titular del teléfono móvil.} \\
\textbf{Tipo} & \textit{String} \\
\textbf{Comentarios} & Este campo es opcional \\
\end{longtable}

\begin{longtable}{p{3cm}  p{12cm}}
\hline
\textbf{Expresión de invariante} & Usuario de teléfono \\
% Descripción 
\textbf{Descripción} & No puede haber Usuarios con el mismo teléfono en el Sistema. \\
\textbf{Comentarios} & - \\
\end{longtable}
