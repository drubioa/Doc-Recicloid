% Tipo que representa SOLICITUDES DE RECOGIDA

\begin{longtable}{p{2.5cm}  p{14cm}}
\caption*{\textbf{TYP-002: SolicitudRecogida}} \\
\hline
\textbf{Versión} & 1.0 (31/03/2014) \\
\textbf{Fuentes} & Apresa21 \\
\textbf{Descripción} & Este tipo de objetos representa  \textit{la solicides de recogidas de muebles y enseres realizadas por los usuarios} \\
\textbf{Supertipo} & - \\
\textbf{Subtipo} & -  \\
\textbf{Comentarios} &- \\
\end{longtable}

% Atributos
\begin{longtable}{p{3cm}  p{12cm}}
\hline
\textbf{Atributo variable} & SolicitudRecogida::fecha \\
\textbf{Descripción} & Este atributo representa  \textit{la fecha en la cual se establece la solicitud de recogida.} \\
\textbf{Tipo} & \textit{String} \\
\textbf{Comentarios} & - \\
\end{longtable}

\begin{longtable}{p{3cm}  p{12cm}}
\hline
\textbf{Atributo variable} & SolicitudRecogida::numeroSolicitud \\
\textbf{Descripción} & Este atributo representa  \textit{el número de solicitud.} \\
\textbf{Tipo} & \textit{Natural} \\
\textbf{Comentarios} & Este atributo es identifica y sirve para contabilizar el número de solicitudes realizadas. \\
\end{longtable}

% Invariantes
\begin{longtable}{p{3cm}  p{12cm}}
\hline
\textbf{Expresión de invariante} & SolicitudRecogida::UnicidadnumeroSolicitud \\
\textbf{Descripción} &  \textit{No puede haber dos solicitudes de recogida con el mismo número} \\
\textbf{Comentarios} & -\\
\end{longtable}
