Tras la entrevista llevada a cabo (véase Anexo Entrevista de elicitación) con algunos de los encargados del sistema actual, gestionado por la empresa Apresa21\cite{apresa21:web} y la consulta de su web, se llega a conocer cómo funciona el sistema actual. A continuación se describe el mismo:

% Información del servicio prestado
\subsubsection*{Servicio ofertado}
El servicio proporcionado a habitantes de la localidad de Puerto Real consiste en gestionar la recogida de muebles o electrodomésticos de tamaño considerable, y su posterior traslado al punto limpio. Dicho servicio es gratuito y coree a cargo del ayuntamiento de Puerto Real. Dada la gratuidad tanto del servicio como de un número de teléfono puesto a disposición de la ciudadanía, se limita el número de peticiones diarias por habitante en tres enseres diarios. Para llevar a cabo dicha tarea la empresa \textit{Apresa 21} dispone de un camión que todos los días realiza una ruta planificada en función de los pedidos realizados y que tiene una capacidad para 25 muebles. \\

% Teléfono disponible
\subsubsection*{Teléfono gratuito}
Actualmente para solicitar la recogida de un mueble o electrodoméstico el habitante de la localidad debe ponerse en contacto con número de teléfono gratuito puesto a disposición para dicha finalidad. En dicha llamada, un teleoperador informará al usuario de las características del servicio ofrecido por el ayuntamiento y solicitará al usuario información sobre los objetos que desea que sean recogidas. También se le solicitará algún identificador su nombre, o si es posible su nombre y apellidos. Una vez se ha identificado al usuario también se requerirán su dirección y un número de teléfono. \\

Una vez el usuario ha descrito que desea depositar, el teleoperador consulta las solicitudes que se han realizado anteriormente y concierta un día para la recogida. Se informa al solicitante de la fecha en la que se recogerán sus objetos, la hora a la que deberá depositarlos y el punto en que se efectuará la recogida por parte del camión. Una vez informado al usuario, se finaliza la llamada telefónica y se registra la solicitud. Este procedimiento es el habitual para estos casos, es un sistema mas costoso, poco automatizado, que requiere de más tiempo. Además requiere que aquellos usuarios interesados dispongan del teléfono y efectúan una llamada telefónica en el horario de atención al público. Además. los usuarios tienen la información únicamente cuando realizan la llamada, posteriormente deben recordar la fecha de recogida, el lugar y toda la información explicada por el teleoperador. No existe un soporte físico donde consultar la recogida, las condiciones y las características. \\

En dichas llamadas además de  tener que explicar al ciudadano todo el procedimiento a seguir, hay que explicar determinadas reglas como por ejemplo el hecho de que deban depositar los enseres a partir de una determinada hora de la noche para evitar posible actos vandálicos; la limitación de 3 muebles por día y por persona. \\

% Funcionamiento interno
\subsubsection	*{Ejecución de la operativa}
Se realizan cada día 25 recogías como máximo (Debido a la capacidad del camión), el conductor del camión dispone de una ruta en la que se contemplan todos los puntos de recogida. Cuando el camión llega a un punto pueden darse dos circunstancias: o bien se localizan los enseres en cuyo caso se colocan en el camión y se pasa al siguiente punto, o bien que haya alguna incidencia en el proceso de recogida. De entre las incidencias más comunes se pasan a enumerar y explicar a continuación: \\

% Incidencias
\begin{enumerate}
\item \textbf{Enseres ausentes o en una ubicación inaccesible: }puede ocurrir que al llegar al punto de recogida el conductor del camión no localice los enseres indicados, o bien se encuentren dentro de una casa o casa puerta a la que no responda nadie. En estos casos el conductor se pone en contacto desde su teléfono móvil con la central, y ésta contacta con vía telefónica con el usuario para resolver la incidencia de forma individual.
\item \textbf{Los objetos depositados difieren de la petición realizada: }este hecho ocurre cuando el conductor del camión encuentra que ya sea el número de objetos depositados y/o su casuística difieren de lo solicitado. Este puede deberse a varios motivos:  los vecinos del solicitante han optado por aprovechar la petición de recogida para depositar (sin previo aviso) sus propios muebles, o bien el usuario por algún motivo ha indicado de manera errónea (intencionada o desintencionadamente) el tipo y/o número de enseres que iba a depositar. Cuando este tipo de incidencia ocurre hay que alterar toda la ruta, aumentar el número de viajes, transbordos e incluso puede que haya que modificar la operativa para dicho día.
\end{enumerate}

% Zonas rurales
Aunque se ha indicado que todos los días se recogen 25 enseres, hay una excepción referida a la zonas geográficas calificada como áreas rurales. En dichas zonas la recogida se realiza un único día a la semana dado que tienen menos demanda de servicio y dispone de espacios en los que almacenar los enseres a la espera de ser recogidos. \\
