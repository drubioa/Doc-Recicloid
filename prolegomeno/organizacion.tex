Todos los roles definidos para la organización del proyecto se han asignado de acuerdo con la metodología \textit{Mobile-D}\cite{AgileMobileD}.
\paragraph{Grupo de cliente}Inicialmente se procederá a identificar al \textit{cliente} del proyecto. Se plantea como finalidad del proyecto: ofrecer una solución a una problemática existente por parte del Ayuntamiento de Puerto Real en lo referido a la recogida de muebles y otros enseres. Se hace necesario el establecer como cliente al responsables del área de sistemas informáticos y a encargado de la gestión logística de limpieza y operaciones de recogida de muebles. Por lo tanto, se establecen como responsables de la comunicación con la entidad cliente al \textit{Analista de Sistema del Ayuntamiento de Puerto Real} y al \textit{Responsable de operativa del grupo Epresa21}. Como vía de contacto se incluye la dirección de correo electrónico del analista de sistemas quien tiene contacto directo con el responsable de operativa. Se definen para cada uno de los responsables de comunicación una serie de funciones:

\begin{itemize}
    \item \textit{Analista de Sistemas: }se dirigirán a él todo tipo de consultas técnicas referidas a la aplicación, servidor, comunicaciones e interfaz gráfica. Se informará de la evolución técnica del proyecto, y se efectuará entrega del producto una vez se encuentre en una versión funcional.
    \item \textit{Responsable de operativa: }todo lo referido a operativas, funcionamiento interno, datos a manejar deberá ser consultado al responsable de operativa. Así mismo se corroboran todos los aspectos de lógica de negocio.
\end{itemize}


% Roles alumno
\paragraph{Roles asociados al alumno autor del TFG}Debido a que dicho proyecto se ha realizado de manera individual, todos estos roles han sido asumidos por el propio alumno autor del proyecto. Se explica a continuación cada una de las finalidades de dichos roles:

\begin{itemize}
    \item \textbf{Equipo de exploración: }grupo encargado de realizada la fase de exploración del proyecto. En este caso la totalidad de la fase de exploración será realizada por el alumno autor del proyecto. 
    \item\textbf{Equipo de proyecto: }equipo de desarrollo del proyecto. Además de las actividades de desarrollo se incluye actividades de arquitectura, testear y recolectar métricas. Todas las tareas anteriormente enumeradas serán llevadas a cabo únicamente por el alumno autor del proyecto.
    \item \textbf{Grupo directivo: }encargado de realizar la toma decisiones referentes al proyecto. Todas las decisiones finales referidas al proyecto, aunque consultadas con el cliente, serán tomadas por el alumno autor del proyecto.
    \item \textbf{Grupo de soporte: }subroles requeridos a lo largo del proyecto. Se contempla la inclusión externa de colaboración para la elaboración de la interfaz gráfica de la app.
    \item \textbf{Grupo de test del sistema: }las pruebas del sistema serán realizadas por el alumno autor del proyecto.
    \item \textbf{Secretario: }se encargará de crear y mantener un listado de las acciones que se van a realizar, y actualizarlos en función del progreso del proyecto. 
\end{itemize}

% Roles tutora
\paragraph{Roles asociados a la tutora del proyecto}siguiente la metodología se asignan a la tutora del proyecto los siguientes roles:
\begin{itemize}
    \item \textbf{Moderador: }la metodología nos indica que es preferible que sea alguien fuera del equipo de proyecto. Se encargará de agrupar los resultados, comprobar el cumplimiento del calendario y gestionar discusiones. Además debe de ser un punt o de unión entre el estado del proyecto y la organización. 
    \item \textbf{Auditor Externo: }persona quién realizará las auditorías y detectará deficiencias. Decidirá si los resultados son válidos y aceptables. El auditor debe tener conocimiento del proyecto. 
\end{itemize}